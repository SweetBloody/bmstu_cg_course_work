\chapter*{ВВЕДЕНИЕ}
\addcontentsline{toc}{chapter}{ВВЕДЕНИЕ}

В настоящее время одной из главных задач компьютерной графики является построение трехмерных изображений. Огромную роль играют программы, позволяющие моделировать 3D объекты, так как прежде чем что-нибудь выпустить в производство, необходимо понять, как будет выглядеть конечный продукт и какой будет последовательность его сборки.

В данной работе было решено создать программу-конструктор, основанную на всем известном конструкторе LEGO, которая позволяла бы создавать трехмерные фигуры из предложенных деталей Лего.

Цель работы --- разработать программу моделирования детского конструктора LEGO. В программе задан набор элементов конструктора, которым пользователь может манипулировать. Из данных элементов можно собирать различные конструкции, выставляя одни элементы на другие. В программе предусмотрено управление положением сцены для предоставления возможности осматривать ее с разных сторон. Также можно установить источник света.

Для реализации поставленной цели следует решить следующие задачи:
\begin{itemize}[label=---]
    \item изучить модели представления объектов и выбрать подходящую;
    \item выбрать алгоритмы для отображения объектов на сцене;
    \item выбрать механизм размещения объектов (при работе с конструктором обычно подразумевается, что размещение объектов происходит так или иначе упорядоченным способом);
    \item выбрать язык программирования и среды разработки;
    \item реализовать выбранные алгоритмы отображения объектов;
    \item реализовать программное обеспечение с пользовательским интерфейсом, позволяющее собирать конструкции из блоков LEGO.
\end{itemize}