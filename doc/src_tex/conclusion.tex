\chapter*{ЗАКЛЮЧЕНИЕ}
\addcontentsline{toc}{chapter}{ЗАКЛЮЧЕНИЕ}

Цель, которая была поставлена в начале курсовой работы, была достигнута: разработана программа моделирования детского конструктора LEGO.

Решены все поставленные задачи:
\begin{itemize}[label=---]
	\item изучены модели представления объектов и выбрана подходящая;
	\item выбраны алгоритмы для отображения объектов на сцене;
    \item выбран механизм размещения объектов;
	\item выбран язык программирования и среды разработки;
    \item реализованы выбранные алгоритмы отображения объектов;
    \item реализовано программное обеспечение с пользовательским интерфейсом, позволяющее собирать конструкции из блоков LEGO.
\end{itemize}

В качестве исследования были проведены замеры времени работы обычного алгоритма с z-буфером и его модификации для построения теней в зависимости от количества деталей на платформе. Исходя из результатов, можно сделать следующие выводы:
\begin{itemize}[label=---]
    \item алгоритм с z-буфером для построения теней в среднем работает в 15 раз медленнее, чем обычная его версия;
    \item прирост времени работы алгоритма для построения теней при добавлении очередной детали LEGO больше в 15 раз, чем у алгоритма без модификаций.
\end{itemize}
